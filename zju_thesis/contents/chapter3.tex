% !TEX root = ../main.tex

\chapter{研究基础}
\section{与本项目有关的研究工作积累和已取得的研究工作成绩}

\section{已具备的实验、资料等条件,尚缺少的实验、资料条件和拟解决的途径}
\chapter{高性能图模型}

\section{kNN图构造}

\section{分布式图模型}
机器学习广泛应用在机器人、数据挖掘、物理和生物学等于多领域中,然而随着万维网的发展以及数据收集技术的改进,现实世界的数据集已经在维度和复杂性上快速增长,所以机器学习算法在一定程度上需要改善和扩展。但是设计和实现高效的秉性机器学习算法是具有挑战性的。现有的高级并行框架(如MapReduce)性能不足,同时MPI等低级工具很难使用。有两种基于图的分布式开发方案,一种是2010年CMU的Select实验室提出了GraphLab框架,一种是基于Spark的用于图和图并行计算的Graph计算框架。
\subsection{GraphLab并行构架}
GraphLab将数据抽象成Graph结构,将算法的执行过程抽象成Gather、Apply、Scatter三个步骤。其并行的核心思想是对顶点的切分。如以下示例。示例中,需要完成对$V_0$邻接顶点的求和计算,串行实现中,V0对其所有的邻接点进行遍历,累加求和。而GraphLab中,将顶点$V_0$进行切分,将$V_0$的 边关系以及对应的邻接点部署在两台处理器上,各台机器上并行进行部分求和运算,然后通过master顶点和mirror顶点的通信完成最终的计算。
\begin{enumerate}
\item Gather阶段。
   工作顶点的边 (可能是所有边,也有可能是入边或者出边)从领接顶点和自身收集数据,记为$gather\_data\_i$,各个边的数据graphlab会求和,记为$sum\_data$。这一阶段对工作顶点、边都是只读的。
\item Apply阶段。
Mirror将gather计算的结果sum_data发送给master顶点,master进行汇总为total。Master利用total和上一步的顶点数据,按照业务需求进行进一步的计算,然后更新master的顶点数据,并同步mirror。Apply阶段中,工作顶点可修改,边不可修改。
\item  Scatter阶段。
工作顶点更新完成之后,更新边上的数据,并通知对其有依赖的邻结顶点更新状态。这scatter过程中,工作顶点只读,边上数据可写。
在执行模型中,graphlab通过控制三个阶段的读写权限来达到互斥的目的。在gather阶段只读,apply对顶点只写,scatter对边只写。并行计算的同步通过master和mirror来实现,mirror相当于每个顶点对外的一个接口人,将复杂的数据通信抽象成顶点的行为。

\end{enumerate}


\section{基于GPU的图模型}